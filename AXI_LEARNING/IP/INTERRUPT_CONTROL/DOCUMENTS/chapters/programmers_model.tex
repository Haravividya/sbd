\chapter{Programmers' Model}
This chapter describes the ZIC Memory-mapped registers and CSRs. It contains the following sections:
\begin{itemize}
    \item \hyperref[sec:about-programmers-model]{About the Programmers' Model}
    \item \hyperref[sec:zic-mem-map-register-description]{ZIC memory-mapped register description}
    \item \hyperref[sec:zic-csr-description]{ZIC CSRs description}
\end{itemize}
\newpage

\section{About the Programmers' Model}
\label{sec:about-programmers-model}
The programmers’ model provides the software interface to the ZIC. This chapter describes the programmers’ model for the ZIC Priority Resolver and Interrupt Request Generator that operates using ZIC memory-mapped registers and CSRs.

The following sections describe the programmers’ model:
\begin{itemize}
    \item \hyperref[subsec:zic-register-mem-map]{ZIC register memory-map}
    \item \hyperref[subsec:zic-csrs-mem-map]{ZIC CSRs memory-map}
\end{itemize}

\subsection{ZIC registers memory-map}
\label{subsec:zic-register-mem-map}
Zilla core has ZIC registers that are accessed by a separate address region. ZIC registers are accessible by the M-mode software running on the Zilla core. The base address of ZIC memory-mapped registers is specified by \texttt{mzicbase} machine CSR.

The ZIC supports up to 48 total interrupt inputs. Table \ref{tab:zic-reg-memory-map} has a list of memory-mapped registers for ZIC.

\vspace{1.25cm}
\begin{table}[H]
    \centering
    \caption{ZIC register memory-map}
    \label{tab:zic-reg-memory-map}
    \vspace{0.75cm}
    \begin{tabular}{l l l l}
    \hline 
        \textbf{Offset} & \textbf{Name} & \textbf{Type} & \textbf{Description}  \\ \hline \hline
        \texttt{0x0000} & \texttt{\hyperref[subsec:zic-cfg]{zic\_cfg}} & R & ZIC Configuration Register \\ \hline
        \texttt{0x0004} & \texttt{\hyperref[subsec:zic-info]{zic\_info}} & R & ZIC Information Register \\ \hline
        \texttt{0x0800} & \texttt{\hyperref[subsec:zic-nxtp-int]{zic\_nxtp\_int}} & R & ZIC Next-Pending Interrupt Register \\ \hline
        \texttt{0x0804} & \texttt{\hyperref[subsec:zic-ack]{zic\_ack}} & R & ZIC Acknowledgement Register \\ \hline
         \texttt{0x0808} & \texttt{\hyperref[subsec:zic-eoi]{zic\_eoi}} & R & ZIC End-of-Interrupt Register \\ \hline 
         & & &  \\ \hline
         \texttt{0x1000} & \texttt{\hyperref[subsec:zic-int-p]{zic\_int\_p\_[0]}} & RW & ZIC Interrupt Pending Register [0] \\ \hline
         \texttt{0x1001} & \texttt{\hyperref[subsec:zic-int-en]{zic\_int\_en\_[0]}} & RW & ZIC Interrupt Enable Register [0]\\ \hline
         \texttt{0x1002} & \texttt{\hyperref[subsec:zic-int-attr]{zic\_int\_attr\_[0]}} & R & ZIC Interrupt Attribute Register [0] \\ \hline
         \texttt{0x1003} & \texttt{\hyperref[subsec:zic-int-ctl]{zic\_int\_ctl\_[0]}} & RW & ZIC Interrupt Control Register [0] \\ \hline
         : & : & : & : \\ \hline
         \texttt{0x10BC} & \texttt{\hyperref[subsec:zic-int-p]{zic\_int\_p\_[47]}} & RW & ZIC Interrupt Pending Register [47]\\ \hline
         \texttt{0x10BD} & \texttt{\hyperref[subsec:zic-int-en]{zic\_int\_en\_[47]}} & RW & ZIC Interrupt Enable Register [47] \\ \hline
         \texttt{0x10BE} & \texttt{\hyperref[subsec:zic-int-attr]{zic\_int\_attr\_[47]}} & R & ZIC Interrupt Attribute Register [47] \\ \hline
         \texttt{0x10BF} & \texttt{\hyperref[subsec:zic-int-ctl]{zic\_int\_ctl\_[47]}} & RW & ZIC Interrupt Control Register [47] \\ \hline
        \end{tabular}
\end{table}

\subsection{ZIC CSRs memory-map}
\label{subsec:zic-csrs-mem-map}
Zilla core has Machine CSRs, some of which are dedicated for interrupt-handling. These CSRs are accessible by ZIC and also by CSR op-codes.

The interrupt-handling CSRs are listed is Table \ref{tab:zic-csr-memory-map}. The changes to existing CSRs and addition of new CSRs for ZIC mode operation are explained in the ZIC CSRs description.

\vspace{1.25cm}
\begin{table}[H]
    \centering
    \caption{ZIC CSRs memory-map}
    \label{tab:zic-csr-memory-map}
    \vspace{0.75cm}
    \begin{tabular}{l l l l}
    \hline 
        \textbf{Address} & \textbf{Name} & \textbf{Type} & \textbf{Description} \\ \hline \hline
        \texttt{} & \texttt{\hyperref[subsec:mstatus]{mstatus}} & RW & Machine Status Register \\ \hline 
        \texttt{} & \texttt{\hyperref[subsec:mtvec]{mtvec}} & RW & Machine Trap Vector Register \\ \hline
        \texttt{} & \texttt{\hyperref[subsec:mtvt]{mtvt}} & RW & Machine Trap Vector Table Register \\ \hline
        \texttt{} & \texttt{\hyperref[subsec:mscratch]{mscratch}} & RW & Machine Scratch Register \\ \hline
        \texttt{} & \texttt{\hyperref[subsec:mepc]{mepc}} & RW & Machine Exception \texttt{pc} Register \\ \hline
        \texttt{} & \texttt{\hyperref[subsec:mcause]{mcause}} & RW & Machine Cause Register \\ \hline
        \texttt{} & \texttt{\hyperref[subsec:mintstatus]{mintstatus}} & RW & Machine Interrupt Status Register \\ \hline
        \texttt{} & \texttt{\hyperref[subsec:mzicbase]{mzicbase}} & RW & Machine ZIC Base Register \\ \hline
        \end{tabular}
\end{table}

\section{ZIC memory-mapped register description}
\label{sec:zic-mem-map-register-description}

\subsection{ZIC Configuration Register, \texttt{zic\_cfg}}
\label{subsec:zic-cfg}

ZIC has a single memory-mapped 8-bit global configuration register, that defines how many privilege modes are supported, how the \texttt{\hyperref[subsec:zic-int-ctl]{zic\_int\_ctl\_[i]}} registers are subdivided into level and priority fields, and whether selective hardware vectoring is supported.

\vspace{0.5cm}
\begin{figure}[H]
    \centering
    \includegraphics[width = 9cm]{images/zic_cfg.png}
    \label{fig:zic_cfg}
\end{figure}
\vspace{0.25cm}

\vspace{0.5cm}
\begin{table}[H]
    \label{tab:zic_cfg}
        \centering
        \begin{tabular}{l l p{8cm} c c}
         \hline 
         \textbf{Bits} & \textbf{Name} & \textbf{Description} & \textbf{Access} & \textbf{Preset}\\ \hline \hline
         [7] & reserved & - & - & 0 \\ \hline
         [6:5] & \texttt{nmbits} & Specifies how many bits are physically implemented in \texttt{zic\_int\_attr\_[i].mode} to represent an input i's privilege mode.
         
         0: M-mode Only
         
         1: M/U Modes
         
         2: M/U/S Modes
         
         3: Reserved & R & 0 \\ \hline
         [4:1] & \texttt{nlbits} & Indicates how many upper bits in \texttt{zic\_int\_ctl\_[i]} are assigned to encode the interrupt levels. 
         
         Valid values are 0-8. & R & 3\\ \hline
         [0] & \texttt{nvbits} & Specifies whether the selective interrupt hardware vectoring feature is implemented or not.
         
         0: selective interrupt hardware vectoring is not implemented.
         
         1: selective interrupt hardware vectoring is implemented. & R & 0\\ \hline
         \end{tabular}
\end{table}

\subsection{ZIC Information Register, \texttt{zic\_info}}
\label{subsec:zic-info}

This is an 8-bit register that holds information such as number of triggers implemented, number of interrupts supported, architecture and implementation version which are useful for debugging.

\vspace{0.5cm}
\begin{figure}[H]
    \centering
    \includegraphics[width = 15.25cm]{images/zic_info.png}
    \label{fig:zic_cfg}
\end{figure}
\vspace{0.25cm}

\vspace{0.5cm}
\begin{table}[H]
    \label{tab:zic_info}
        \centering
        \begin{tabular}{l l p{6cm} c c}
         \hline 
         \textbf{Bits} & \textbf{Name} & \textbf{Description} & \textbf{Access} & \textbf{Preset}\\ \hline \hline
         [31] & reserved & - & - & 0 \\ \hline
         [30:25] & \texttt{num\_trigger} & Specifies the number of maximum interrupt triggers supported in this implementation.
         
         Valid values are 0-32. & R & 0\\ \hline
         [24:21] & \texttt{zic\_int\_ctl\_bits\footnotemark} & Specifies how many hardware bits are implemented in the \texttt{zic\_int\_ctl} registers. 

         Valid values are 0-8. & R & 6\\ \hline
         [20:17] & \texttt{arch\_version} & Specifies Architecture Version. & R & 0\\ \hline
         [16:13] & \texttt{impl\_version} & Specifies Implementation Version. & R & 0 \\ \hline
         [12:10] & \texttt{num\_interrupts} & Specifies the actual number of maximum interrupt inputs supported in this implementation & R & 47 \\ \hline
         \end{tabular}
\end{table}

\footnotetext{The implemented bits are kept left-justified in the most significant bits of each 8-bit \texttt{zic\_int\_ctl\_[i]} register, with the lower unimplemented bits treated as hardwired to 1.}

\subsection{ZIC Next-Pending Interrupt Register, \texttt{zic\_nxtp\_int}}
\label{subsec:zic-nxtp-int}
This is an 8-bit register that holds the level of highest pending interrupt.

\vspace{0.5cm}
\begin{figure}[H]
    \centering
    \includegraphics[width = 9cm]{images/zic_nxtp_int.png}
    \label{fig:zic_nxtp_int}
\end{figure}
\vspace{0.25cm}

\vspace{0.5cm}
\begin{table}[H]
    \label{tab:zic_nxtp_int}
        \centering
        \begin{tabular}{l l p{7cm} c c}
         \hline 
         \textbf{Bits} & \textbf{Name} & \textbf{Description} & \textbf{Access} & \textbf{Preset}\\ \hline \hline
         [7:0] & \texttt{zic\_nxtp\_int} & This register will be written with the highest level of pending interrupt after priority resolution. & RW & 0\\ \hline
        \end{tabular}
\end{table}
\vspace{0.5cm}

Priority Resolver performs a comparison of all the pending interrupts by comparing the local buffers to check which interrupt has the highest level-priority value. At the end of the comparison, the level-priority value of that interrupt is written to this register.

Interrupt Request Generator reads this register and compares it with the current interrupt level form \texttt{\hyperref[subsec:mintstatus]{mintstatus}} and decides whether the highest pending interrupt has sufficient priority.

\subsection{ZIC Acknowledgement Register, \texttt{zic\_ack}}
\label{subsec:zic-ack}
This is an 8-bit register that holds the ID value of the requested interrupt, before the acknowledgment from the processor.

\vspace{0.5cm}
\begin{figure}[H]
    \centering
    \includegraphics[width = 9cm]{images/zic_ack.png}
    \label{fig:zic_ack}
\end{figure}
\vspace{0.25cm}

\vspace{0.5cm}
\begin{table}[H]
    \label{tab:zic_ack}
        \centering
        \begin{tabular}{l l p{8cm} c c}
         \hline 
         \textbf{Bits} & \textbf{Name} & \textbf{Description} & \textbf{Access} & \textbf{Preset}\\ \hline \hline
         [7:0] & \texttt{zic\_ack} & This register will be read by the processor to serve the interrupt. & RW & 0\\ \hline
        \end{tabular}
\end{table}
\vspace{0.5cm}

 When the processor reads this register, Interrupt Request Generator recognizes this read, clears the pending bit in \texttt{zic\_int\_p\_[i]}, and writes the level-priority value into \texttt{mintstatus} CSR.

\subsection{ZIC End-of-Interrupt Register, \texttt{zic\_eoi}}
\label{subsec:zic-eoi}
This is an 8-bit register that holds the ID value of interrupt currently being served.

\vspace{0.5cm}
\begin{figure}[H]
    \centering
    \includegraphics[width = 9cm]{images/zic_eoi.png}
    \label{fig:zic_eoi}
\end{figure}
\vspace{0.25cm}

\vspace{0.5cm}
\begin{table}[H]
    \label{tab:zic_eoi}
        \centering
        \begin{tabular}{l l p{8cm} c c}
         \hline 
         \textbf{Bits} & \textbf{Name} & \textbf{Description} & \textbf{Access} & \textbf{Preset}\\ \hline \hline
         [7:0] & \texttt{zic\_eoi} & The processor writes the ID of the interrupt into this register once it completes executing the ISR. & RW & 0\\ \hline
        \end{tabular}
\end{table}
\vspace{0.5cm}

\subsection{ZIC Interrupt Pending Registers, \texttt{zic\_int\_p\_[i]}}
\label{subsec:zic-int-p}
Every interrupt supported by the platform has a dedicated interrupt pending register and occupies one byte in the memory map for ease of access. The pending bit is located in bit 0 of the byte.

\vspace{0.5cm}
\begin{figure}[H]
    \centering
    \includegraphics[width = 9cm]{images/zic_int_p.png}
    \label{fig:zic_int_p}
\end{figure}
\vspace{0.25cm}

\vspace{0.5cm}
\begin{table}[H]
    \label{tab:zic_int_p}
        \centering
        \begin{tabular}{l l p{8cm} c c}
         \hline 
         \textbf{Bits} & \textbf{Name} & \textbf{Description} & \textbf{Access} & \textbf{Preset} \\ \hline \hline
         [7:1] & reserved & - & - & 0 \\ \hline
         [0] & p & 0: Interrupt is requested from the source
         
         1: Interrupt is de-asserted from the source
 & RW & 0\\ \hline
        \end{tabular}
\end{table}
\vspace{0.5cm}

If the interrupt line is enabled globally and is asserted from the source, the Priority Resolver sets the corresponding interrupt pending register bit.

Priority Resolver will clear this bit when the End of Interrupt signal is detected.

\subsection{ZIC Interrupt Enable Registers, \texttt{zic\_int\_en\_[i]}}
\label{subsec:zic-int-en}

Every interrupt supported by the platform has a dedicated interrupt enable register and occupies one byte in the memory map for ease of access. This enable bit is located in bit 0 of the byte.

\vspace{0.5cm}
\begin{figure}[H]
    \centering
    \includegraphics[width = 9cm]{images/zic_int_en.png}
    \label{fig:zic_int_en}
\end{figure}
\vspace{0.25cm}

\vspace{0.5cm}
\begin{table}[H]
    \label{tab:zic_int_en}
        \centering
        \begin{tabular}{l l p{8cm} c c}
         \hline 
         \textbf{Bits} & \textbf{Name} & \textbf{Description} & \textbf{Access} & \textbf{Preset}\\ \hline \hline
         [7:1] & reserved & - & - & 0 \\ \hline
         [0] & en & 0: Interrupt is disabled
         
         1: Interrupt is enabled & RW & 0\\ \hline
        \end{tabular}
\end{table}
\vspace{0.5cm}

\subsection{ZIC Interrupt Attribute Registers, \texttt{zic\_int\_attr\_[i]}}
\label{subsec:zic-int-attr}

This is an 8-bit register that specify various attributes such as privilege mode, trigger type and hardware vectoring for each interrupt.

\vspace{0.5cm}
\begin{figure}[H]
    \centering
    \includegraphics[width = 9cm]{images/zic_int_attr.png}
    \label{fig:zic_int_attr}
\end{figure}
\vspace{0.5cm}

\vspace{0.5cm}
\begin{table}[H]
    \label{tab:zic_int_attr}
        \centering
        \begin{tabular}{l l p{8cm} c c}
         \hline 
         \textbf{Bits} & \textbf{Name} & \textbf{Description} & \textbf{Access} & \textbf{Preset}\\ \hline \hline
         [7:6] & mode & Specifies which privilege mode this interrupt operates in
         
         11: machine mode 
         
         01: supervisor mode
         
         00: user mode & R & 3 \\ \hline
         [5:3] & reserved & - & - & 0 \\ \hline
         [2:1] & trig & Specifies the trigger type and polarity for each interrupt
         
         00: Positive level-triggered
         
         01: Positive edge-triggered
         
         10: Negative level-triggered
         
         11: Negative edge-triggered & R & \\ \hline
         [0] & shv & Selective Hardware Vectoring & R & \\ \hline
        \end{tabular}
\end{table}
\vspace{0.5cm}

\subsection{ZIC Interrupt Control Registers, \texttt{zic\_int\_ctl\_[i]}}
\label{subsec:zic-int-ctl}

This is an 8-bit memory-mapped register to control the interrupt pre-emption (nesting) using levels and priorities. The number of bits implemented in this register is specified by a fixed parameter \texttt{zic\_int\_ctl\_bits} (in \texttt{zic\_info}), which has a value between 0 to 8. 

The implemented bits are kept left-justified in the most significant bits of each 8-bit \texttt{zic\_int\_ctl\_[i]} register, with the lower unimplemented bits treated as hardwired to 1.

These control bits are interpreted as level and priority according to the setting in the ZIC
Configuration register (\texttt{zic\_cfg.nlbits}).

\vspace{0.5cm}
\begin{figure}[H]
    \centering
    \includegraphics[width = 9cm]{images/zic_int_ctl.png}
    \label{fig:zic_int_ctl}
\end{figure}
\vspace{0.25cm}

\vspace{0.5cm}
\begin{table}[H]
    \label{tab:zic_int_ctl}
        \centering
        \begin{tabular}{l l p{8cm} c c}
         \hline 
         \textbf{Bits} & \textbf{Name} & \textbf{Description} & \textbf{Access} & \textbf{Preset}\\ \hline \hline
         [7:0] & \texttt{zic\_int\_ctl} & Specifies the Level and Priority of the interrupt & RW & 0\\ \hline
        \end{tabular}
\end{table}
\vspace{0.5cm}

When an interrupt is enabled and the corresponding pending bit is set, Priority Resolver loads the level-priority value into a local buffers, holding the pending bit and level-priority value.

\section{ZIC CSRs Description}
\label{sec:zic-csr-description}

\subsection{Machine Status Register, \texttt{mstatus}}
\label{subsec:mstatus}

This is a 64-bit CSR that keeps track of and controls the hart's current operating state.

\texttt{mie} is hardwired to zero in ZIC mode, replaced by separate memory-mapped interrupt enable registers (\texttt{zic\_int\_en\_[i]}).

\texttt{mip} is hardwired to zero in ZIC mode, replaced by separate memory-mapped interrupt pending registers (\texttt{zic\_int\_p\_[i]}).

Writes to \texttt{mie/mip} will be ignored and will not trap (i.e., no access faults). 

\subsection{Machine Trap Vector Register, \texttt{mtvec}}
\label{subsec:mtvec}

This is a 64-bit CSR that holds trap vector configuration, consisting of a vector base address (BASE) and a vector mode (MODE).

\vspace{0.5cm}
\begin{figure}[H]
    \centering
    \includegraphics[width = 15.25cm]{images/csr_mtvec.png}
    \label{fig:csr_mtvec}
\end{figure}
\vspace{0.25cm}

\vspace{0.5cm}
\begin{table}[H]
    \label{tab:csr_mtvec}
        \centering
        \begin{tabular}{l l p{8cm} c c}
         \hline 
         \textbf{Bits} & \textbf{Name} & \textbf{Description} & \textbf{Access} & \textbf{Preset}\\ \hline \hline
         [63:2] & BASE & Specifies the base address & WARL & 0 \\ \hline
         [1:0] & MODE & Specifies the mode of operation
         
         0: Direct - All exceptions are in Direct mode 
         
         \hspace{1cm} \texttt{pc} = BASE.
         
         1: Vectored - All asynchronous interrupts are in Vectored mode
         
         \hspace{1cm} \texttt{pc} = BASE + (4 $\times$ cause).
         
         $\geq$ 2: Reserved & WARL & 0\\ \hline
        \end{tabular}
\end{table}
\vspace{0.5cm}

\subsection{Machine Trap Vector Table Register, \texttt{mtvt}}
\label{subsec:mtvt}

This is a 64-bit CSR that holds the base address of the trap vector table.

\vspace{0.5cm}
\begin{figure}[H]
    \centering
    \includegraphics[width = 15.25cm]{images/csr_mtvt.png}
    \label{fig:csr_mtvt}
\end{figure}
\vspace{0.25cm}

It is aligned on a 64-byte or greater power-of-two boundary. 

The actual alignment can be determined by writing ones to the low-order bits then reading them back. Values other than 0 in the low 6 bits of \texttt{xtvt} are reserved.

\subsection{Machine Scratch Register, \texttt{mscratch}}
\label{subsec:mscratch}

This is a 64-bit CSR dedicated for use by machine mode.

\vspace{0.5cm}
\begin{figure}[H]
    \centering
    \includegraphics[width = 15.25cm]{images/csr_mscratch.png}
    \label{fig:csr_mscratch}
\end{figure}
\vspace{0.25cm}

Typically, it is used to hold a pointer to a machine-mode hart-local context space and swapped with a user register upon entry to an M-mode trap handler.

\subsection{Machine Exception \texttt{pc} Register, \texttt{mepc}}
\label{subsec:mepc}

This is a 64-bit register that stores the \texttt{pc} value when normal execution is interrupted.

\vspace{0.5cm}
\begin{figure}[H]
    \centering
    \includegraphics[width = 15.25cm]{images/csr_mepc.png}
    \label{fig:csr_mepc}
\end{figure}
\vspace{0.25cm}

\subsection{Machine Cause Register, \texttt{mcause}}
\label{subsec:mcause}

This is a 64-bit CSR that holds the information regarding cause of trap and additional book-keeping like previous interrupt status. When a trap is taken into M-mode, \texttt{mcause} is written with a code indicating the event that caused the trap.

Otherwise, mcause is never written by the implementation, though it may be explicitly written by software.

\vspace{0.5cm}
\begin{figure}[H]
    \centering
    \includegraphics[width = 15.25cm]{images/csr_mcause.png}
    \label{fig:csr_mcause}
\end{figure}
\vspace{0.25cm}

\vspace{0.5cm}
\begin{table}[H]
    \label{tab:csr_mcause}
        \centering
        \begin{tabular}{l l p{8cm} c c}
         \hline 
         \textbf{Bits} & \textbf{Name} & \textbf{Description} & \textbf{Access} & \textbf{Preset} \\ \hline \hline
         [63] & interrupt & Specifies the type of trap.
         
         0: Exception
         
         1: Interrupt & RW & 0\\ \hline
         [62:31] & reserved & - & - & 0\\ \hline
         [30] & minhv & Specifies the status of the Hardware Vectoring
         
         0: End of successful hardware vectoring
         
         1: Start of hardware vectoring & R & 0\\ \hline
         [29:28] & mpp & Previous privilege mode
         
         (same as mstatus.mpp) & R & 3\\ \hline
         [27] & mpie & Previous Interrupt Enable & R & 0\\ \hline
         [26:24] & reserved & - & - & 0\\ \hline
         [23:16] & mpil & Previous interrupt level & R \\ \hline
         [15:12] & reserved & - & - & 0\\ \hline
         [11:0] & excode & Exception/interrupt code & WARL & 0\\ \hline
        \end{tabular}
\end{table}
\vspace{0.5cm}

\subsection{Machine Interrupt Status, \texttt{mintstatus}}
\label{subsec:mintstatus}

This is a 64-bit CSR that holds the active interrupt level for machine privilege mode. The primary reason to expose these fields is to support debugging.

\vspace{0.5cm}
\begin{figure}[H]
    \centering
    \includegraphics[width = 15.25cm]{images/csr_mintstatus.png}
    \label{fig:csr_mintstatus}
\end{figure}
\vspace{0.25cm}

\vspace{0.5cm}
\begin{table}[H]
    \label{tab:csr_mintstatus}
        \centering
        \begin{tabular}{l l p{8cm} c c}
         \hline 
         \textbf{Bits} & \textbf{Name} & \textbf{Description} & \textbf{Access} & \textbf{Preset}\\ \hline \hline
         [63:32] & reserved & - & - & 0\\ \hline
         [31:24] & mil & Specifies Machine-mode active Interrupt Level & R & 0\\ \hline
         [23:16] & reserved & - & - & 0\\ \hline
         [15:8] & sil & Specifies Supervisor-mode active Interrupt Level
         
         This field is harwired to 0. & R & 0\\ \hline
         [7:0] & uil & Specifies User-mode active Interrupt Level
         
         This field is harwired to 0. & R & 0\\ \hline
        \end{tabular}
\end{table}
\vspace{0.5cm}

Interrupt Request Generator reads \texttt{mil} value and compares it with the \linebreak \texttt{zic\_nxtp\_int} value to decide whether the next pending interrupt has sufficient level priority. 

If the new interrupt has sufficient priority, then the Interrupt Request Generator generates interrupt requests to the core processor and writes corresponding Interrupt ID into the \texttt{zic\_ack} register.


\subsection{Machine ZIC Base Register, \texttt{mzicbase}}
\label{subsec:mzicbase}

This is a 64 - bit  CSR that stores the base address of the memory-mapped ZIC registers.

\vspace{0.5cm}
\begin{figure}[H]
    \centering
    \includegraphics[width = 15.25cm]{images/csr_mzicbase.png}
    \label{fig:csr_mzicbase}
\end{figure}
\vspace{0.25cm}

\vspace{0.5cm}
\begin{table}[H]
    \label{tab:csr_mzicbase}
        \centering
        \begin{tabular}{l l p{8cm} c c}
         \hline 
         \textbf{Bits} & \textbf{Name} & \textbf{Description} & \textbf{Access} & \textbf{Preset}\\ \hline \hline
         [63:0] & mzicbase & Base address for ZIC memory-mapped register & R & 0\\ \hline
        \end{tabular}
\end{table}
\vspace{0.5cm}
